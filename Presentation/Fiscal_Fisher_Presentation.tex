%%%% MEASURING TREASURY DEBT AND MARKET DEPTH %%%%
%%%%%%%%%%%% CONFERENCE PRESENTATION %%%%%%%%%%%%%


%:PREAMBLE

% << Handout Slide Version >> %
\documentclass[11pt, handout, aspectratio=169]{beamer}

% << Class Slide Version >> % 
%\documentclass[11pt, aspectratio=169]{beamer}


%:	THEMES
\usetheme{Madrid}
\usecolortheme{default}

%:	DEFINING COLORS
\definecolor{WeberPurple}{rgb}{.286, .137, .396}
\definecolor{WeberGray}{rgb}{.294, .286, .271}
\definecolor{WeberSecondary}{rgb}{.639, .569, .694}
\definecolor{blue}{RGB}{3, 59, 150}%{0,114,178}
\definecolor{red}{RGB}{213,94,0}
\definecolor{yellow}{RGB}{240,228,66}
\definecolor{green}{RGB}{0,158,115}
\definecolor{greyblue}{RGB}{178, 204, 247}
\definecolor{quizred}{RGB}{242, 52, 48}

%:	RESETTING TEMPLATE COLORS 
\setbeamercolor*{palette primary}{use=structure,fg=white,bg=WeberPurple}
\setbeamercolor*{palette secondary}{use=structure,fg=white,bg=WeberSecondary}
\setbeamercolor*{palette tertiary}{use=structure,fg=white,bg=WeberGray}
\setbeamercolor{local structure}{fg=WeberPurple}
\setbeamercolor{section in toc}{fg=WeberPurple}
\setbeamercolor{subsection in toc}{fg=WeberGray}
\setbeamercolor{block title}{bg=WeberPurple,fg=white}
\setbeamercolor{block body}{bg=WeberSecondary,fg=black}

%:	PACKAGES
\usepackage{amsmath}
\usepackage{amsthm}
\usepackage{amssymb}
\usepackage[retainorgcmds]{IEEEtrantools}
\usepackage{graphicx}
\usepackage{multicol}
\usepackage{multirow}
\usepackage[english]{babel}
\usepackage{color} 
\usepackage[labelformat=empty]{caption}
\usepackage{threeparttable}
\usepackage{dcolumn}
\newcolumntype{d}{D{.}{.}{3.4}}
\newcolumntype{v}{D{.}{.}{3.2}}
\newcolumntype{R}{>{\raggedleft\arraybackslash}X}
\usepackage{subfigure}
%\usepackage{appendixnumberbeamer}
\usepackage{epsfig}
\setbeamertemplate{bibliography item}{$\cdot$}
\usepackage{xfrac}
\usepackage[all]{xy}
\usepackage{tikz}
\usetikzlibrary{calc,intersections,positioning,patterns,decorations.pathreplacing}
\usepackage{wasysym}
\usepackage{tabularx}
\usepackage{booktabs}
\usepackage{eurosym}
\usepackage{framed} 	% For the career inserts
%\usepackage{epstopdf}
%\beamertemplatenavigationsymbolsempty

%:	ADJUSTING THE TABLE OF CONTENTS
\mode<presentation>{\AtBeginSection[]{
	\begin{frame}
    \frametitle{Roadmap of Today's Lecture}
    \tableofcontents[currentsection]
	\end{frame}
	}
	}
\makeatother

\mode<presentation>{
	\setcounter{tocdepth}{2}
}

%:	ADJUSTING ITMEIZE/ENUMERATE ENVIRONMENTS
\newenvironment{wideitemize}{\itemize\addtolength{\itemsep}{10pt}}{\enditemize} 
\newenvironment{wideenumerate}{\enumerate\addtolength{\itemsep}{10pt}}{\endenumerate}

%:	ADDING NEW ENVIRONMENTS
\newcolumntype{d}[1]{D{.}{.}{#1}}

%:TITLE MATERIAL
\title[Measuring Treasury Debt and Depth]{Measuring Treasury Debt and Market Depth}
\author[Keinsley]{Dr. Andrew Keinsley}
\institute[WSU]{Weber State University}
\date{November, 2023}

%:BEGIN DOCUMENT
\begin{document}

\frame{\maketitle}

%:	INTRODUCTION

\begin{frame}
\frametitle{The Basics}
\begin{columns}[t]
	\begin{column}{0.49\textwidth}
		\begin{wideitemize}
			\item US fiscal debt has come back into sharp focus recently
			\begin{itemize}
				\item COVID-19
				\item Industrial policy
				\item Inflation
				\item Rising interest rates
			\end{itemize}
			\item Traditional view of the UST market focuses on size, not depth 
		\end{wideitemize}	
	\end{column}
	\hfill
	\begin{column}{0.49\textwidth}
		\begin{wideitemize}
			\item Contributions
			\begin{wideenumerate}
				\item Simple sum of USTs is incorrect
				\item Derivation of user cost of USTs
				\item Creation of index to track true aggregate
				\item Value of USTs directly impacts fiscal sustainability
			\end{wideenumerate}
		\end{wideitemize}
	\end{column}
\end{columns}
\end{frame}

%:	QUALITIES OF THE UST MARKET

\begin{frame}
	\frametitle{USTs are Imperfect Substitutes}
	\vspace{-2em}
	\begin{columns}[t]
		\begin{column}{0.49\textwidth}
			\begin{center}
				\Large \textcolor{WeberPurple}{Literature}
			\end{center} \vspace{-.2in}
			{\color{WeberPurple}\rule{\linewidth}{2pt}}
			\begin{wideitemize}
				\item Krishnamurthy and Vissing-Jorgensen (2012, 2013)
				\item Nagel (2016) \\ $\vdots$
				\item All: the various maturities/types of UTSs have different attributes/purposes
			\end{wideitemize}
		\end{column}
		\hfill
		\begin{column}{0.49\textwidth}
			\begin{center}
				\Large \textcolor{WeberPurple}{Findings}
			\end{center} \vspace{-.2in}
			{\color{WeberPurple}\rule{\linewidth}{2pt}}
			\begin{wideitemize}
				\item Extension of Amihud and Mendelson (1991)
				\begin{itemize}
					\item Match securities that mature within one day of each other
					\item Regress YTM spreads against a variety of factors
				\end{itemize}
				\item Bills are a liquidity hedge
				\item Bonds are a savings vehicle
				\item They should not be linearly aggregated 
			\end{wideitemize}
		\end{column}
	\end{columns}
\end{frame}

\begin{frame}
	% \frametitle{Empirical Results}
	\begin{table}[p]
		\centering
		\setlength{\tabcolsep}{15pt}
		\renewcommand{\arraystretch}{1.2}
		% \caption{Relative Liquidity and Yield to Maturity$^a$} \vspace{1em}
		\label{tab:Relative_Liquidity}
		\resizebox{\textwidth}{!}{
		\begin{threeparttable}
			\begin{tabular}{l d{2.4} d{2.4} d{2.4}} \toprule
											& \multicolumn{1}{c}{Notes--Bills}  	& \multicolumn{1}{c}{Bonds--Notes} 	& \multicolumn{1}{c}{Bonds--Bills}	\\ \midrule
				Relative Bid-Ask Spread 	& 0.4163^{**}								& 0.0981							& -0.1424^{**}		 					\\ [-0.5em]
											& (0.173)									& (0.074)							& (0.074)	 						\\
				Coupon Rate Spread 			& 0.0210^{***}								& 0.0170^{***}						& 0.0076 							\\ [-0.5em]
											& (0.001)									& (0.005)							& (0.013)	 						\\
				% Months to Maturity 			& -0.0152^{***} 							& 									& -0.1028^{***}								 	\\ [-0.5em]
				% 							& (0.002)									&									& (0.018)									\\
				% Years to Maturity 			& 											& -0.0049^{***}						& 			 					\\ [-0.5em]
				% 							& 											& (0.001)							& 		 						\\
				% Year Dummy 					& 0.0027^{***}								& -0.0079^{***}						& -0.0079 			\\ [-0.5em]
				% 							& (0.001) 									& (0.001)							& (0.001)	 		\\ 
				10y-2y Spread 				& 0.0234^{***}								& -0.0124^{***}						& -0.0484 					\\ [-0.5em]
											& (0.003) 									& (0.002)							& (0.033)	 						\\ \midrule
				% Constant					& 0.0076 									& -0.0554^{**}						& -0.3244^{**} 						\\ [-0.5em]
				% 							& (0.010)									& (0.030) 							& (0.154)  							\\\midrule
				Observations				& \multicolumn{1}{c}{2250}					& \multicolumn{1}{c}{7430}			& \multicolumn{1}{c}{78} 			\\
				R-Squared					& 0.207										& 0.505								& 0.399			 					\\
				F-statistic 				& 99.48										& 207.2								& 20.19			 					\\ \bottomrule 
			\end{tabular}
			% \begin{tablenotes}
			% 	\item[a] \footnotesize{The dependent variable is the yield-to-maturity spread between securities matched by thier respective days to maturity. Standard errors are heteroskedasticity robust (HC1). Designations ***, **, and * represent results that are statistivally significant at the one, five, and ten percent levels, respecitivly.}
			% 	\item[b] \footnotesize{Analyses considering Treasury bills are limited to those securities with six months to maturity or less. This ensures that both securities have no other payments remaining until the maturity date.}
			% \end{tablenotes}
		\end{threeparttable}}
	\end{table}
\end{frame}

\end{document}